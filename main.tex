\PassOptionsToPackage{unicode}{hyperref}
\PassOptionsToPackage{hyphens}{url}
%
\documentclass[superscriptaddress, 12pt]{revtex4-2}%twocolumn
\usepackage{amsmath,amssymb}
\usepackage{iftex}
% Use upquote if available, for straight quotes in verbatim environments
\usepackage{xcolor}
\usepackage{subcaption}
%\usepackage{longtable}
\usepackage{booktabs,array}
\usepackage{multirow}
\usepackage{calc} % for calculating minipage widths
% Correct order of tables after \paragraph or \subparagraph
\usepackage{etoolbox}
\usepackage{footnote}
\usepackage{graphicx}
\usepackage{bookmark}
%\usepackage{babel}
\IfFileExists{xurl.sty}{\usepackage{xurl}}{} % add URL line breaks if available
\urlstyle{same}
\hypersetup{
  hidelinks,
  pdfcreator={LaTeX via pandoc}}
\usepackage{textgreek}
\usepackage{makecell}
\usepackage{multirow}
\usepackage{booktabs}
\usepackage[LGR,T1]{fontenc}
\usepackage[utf8]{inputenc}
\usepackage{textalpha}
\usepackage{ulem}
\usepackage{textcomp}
\usepackage{float}
\date{}


\newcommand{\MFnew}[1]{{\color{purple} #1}~}
\newcommand{\MFnewMissRef}{\MFnew{[Missing Ref]}}
\newcommand{\MFmarginpar}[3]{\MFnew{#1} \marginpar{#2 \tiny \MFnew{#1 #3}}}
\newcommand{\fecrnominal}{Fe\textsubscript{16}Cr\textsubscript{14}}

\newcommand{\ijklm}{ijklm}

\newcommand{\siteA}{2\textit{a}}
\newcommand{\siteB}{4\textit{f}}
\newcommand{\siteC}{8\textit{i\textsubscript{1}}}
\newcommand{\siteD}{8\textit{i\textsubscript{2}}}
\newcommand{\siteE}{8\textit{j}}

\raggedbottom

\begin{document}

\title{Influence of partial occupancy on the elastic tensor for FeCr \textsigma - phase.}

\author{Mariano Forti} \email{mariano.forti@icams.rub.de}
\affiliation{Interdisciplinary centre for advanced materials simulations (ICAMS), Ruhr Universität Bochum, Germany}

\author{Guillaume Laplanche} 
\affiliation{Ruhr Universität Bochum}

\author{Wenhao Zhang}
\affiliation{CNRS-Saint-Gobain-NIMS, IRL 3629, Laboratory for Innovative Key Materials and Structures (LINK), 1-1 Namiki, Tsukuba, 305-0044, Ibaraki, Japan}

\author{Thomas Hammerschmidt}
\affiliation{Interdisciplinary cetre for advanced materials simulations (ICAMS), Ruhr Universität Bochum, Germany}

\date{\today}

\maketitle

\section{Introduction}

The \textsigma~ phase is a common intermetallic compound commonly precipitating at high temperatures in a wide variety of structural alloys, like in stainless steels \MFnewMissRef, Co based superalloys \MFnewMissRef wroght superalloys \MFnewMissRef or high entropy alloys \MFnewMissRef. 
Consequences of its precipitation are usually detrimental for both mechanical properties and corrosion resistance \MFnewMissRef.
Although in technological applications this phase appear with multi-component compositions, there are a handfull of binary systems showing \textsigma--phase precipitation that can be taken as a basis to study its properties and which also give rise to commercial alloys with techological applications.

The Fe--Cr system is an important alloy system with applications in structural and high-temperature environments.
Here, the \textsigma-phase can form at Cr concentrations between 45 and 50\%at and temperatures ranging from 550 to 800\textdegree ~\cite{laplanche_phase_2018} significantly influencing the mechanical properties.
Specifically, \textsigma-phase precipitation contributes to embrittlement and reduces corrosion resistance {\color{red}  [ref 2 detrimental sigma] }.
The \textsigma-phase unit cell is shown in \autoref{fig:sigmaUnitCell}.
This is an intermetallic compound with a tetragonal structure (space group \textit{P4\textsubscript{2}/mnm}) and exhibits complex atomic arrangements due to site-specific occupancies.
\MFnew{This gives the \textsigma-phase strong long range order.}
Understanding its properties is essential for predicting its mechanical behavior and phase stability.
There are different approaches to similar problems in atomistic simulations, based for instance on Density Functional Theory (DFT).
Vesti et. al. ~\cite{vesti_ab-initio_2023} utilize a  sublattice model later averaged under the compound energy formalism \MFnew{[missing ref]} to interpolate the properties of a W-Re \textsigma-phase.
Kabliman et. al.\cite{kabliman_ab_2012}
use a mean field approximation to study several properties of the \textsigma-phase in other binary systems, also based on full occupancy of the sublattices.
\MFnew{This methodologies were deviced several years ago, when a full study in the composition and configuration space was not plausible0}.
\MFnew{Their underlying intention was to approximate a disordered structure with a finite set of oredered unit cells}.
These approximations may overlook local atomic configuration effects.
In this work, we investigate how explicit consideration of partial occupancy influences the elastic properties of the \textsigma-phase through first-principles and machine learning interatomic potentials calculations.
We focus on a Fe\textsubscript{16}Cr\textsubscript{14} unit cell which can be modelled with nominal and on-site compositions in agreement with experimental observations.

\begin{figure}
  \subcaptionbox{\protect\label{fig:sigmaUnitCell001}}{\includegraphics[height=4cm]{Figure_SigmaPhase001.png}}
  \subcaptionbox{\protect\label{fig:sigmaUnitCell100}}{\includegraphics[height=4cm]{example-image-b}}
  \caption{\protect\label{fig:introduction}
    (\subref{fig:sigmaUnitCell001}) (001) and (\subref{fig:sigmaUnitCell100}) (100) views of the \textsigma-phase unit cell.
    Wyckoff sites are encoded in colors: 2\textit{a} is light blue,
    4\textit{f} in grey, 8\textit{i} in green, 8\textit{i'} in gold, 8\textit{j} in blue.
  }
\end{figure}

\section{Methodology}

We employ first-principles DFT calculations to determine the elastic constants of the Fe-Cr \textsigma-phase. 
First, we study the change of elastic constants with composition under a full site occupancy approximation.
Then, we focus on a nominal composition of Fe\textsubscript{16}Cr\textsubscript{14} (48\%at. Cr) to study the effect of partial occupancy explicitly.
These calculations are conducted using the Vienna Ab initio Simulation Package (VASP) ~\cite{Hafner_vasp}, utilizing the Perdew-Burke-Ernzerhof (PBE) ~\cite{Perdew1996} exchange-correlation functional and the Projector Augmented Wave (PAW) method ~\cite{Bloch1994, kresse_ultrasoft_1999}.
\MFnew{
\paragraph{Magnetic ordering} The \textsigma-phase in the Fe-Cr is stable at high temperatures and and is observed paramagnetic at both high and room temperatures after quenching \MFnewMissRef. 
Therefore, we calculate elastic constants from non spin polarized calculations only, although the influence of magnetic transitions in mechanical properties will be studied later. 
}

\paragraph{Full occupancy}\label{methodology:fo} First, we calculate the properties of the \textsigma-phase using a sublattice model.
This means, that the lattice positions of a WS are occupied either by Cr or Fe.
This way we calculate  2\textsuperscript{5} = 32 ordered structures. 
These so-called end-member configurations can be used to find the optimal site occupancy and Gibbs free energy at any finite temperature for any nominal Fe (or Cr) composition $x$ using the compound energy formalism (CEF) ~\cite{dupin_using_2006, crivello_first_2009, zhang_prediction_2025}.
Here, the occupancy of an element $i$ in a sublattice $s$ is defined in terms of the number of atoms $N_i ^s$ in the sublattice,

\begin{equation}\label{eqn:occupancy}
  y_i ^s = \dfrac{N_i ^s}{\sum _j N_j ^s}
\end{equation}

\noindent and can be calculated from an optimization problem which minimizes the weighted Gibbs free energy \cite{zhang_prediction_2025}. 
In this approach, the Gibbs free energy is usually derived from a Gorsky-Bragg-Williams approximation without excess terms\cite{gorsky_rontgenographische_1928,bragg_effect_1997,chizhikov_gorsky-bragg-williams_2000}.
In particular for \textsigma-phase,

\begin{align}\label{eqn:CEF}
  \begin{split}
    \Delta G(x, T) &= \Delta H(x, T) - T S(x, T) \\
		   &= \min _{ y_i ^{2a}  \dots y_m ^{8i} } 
		   \left[  
		     \sum_{i j k l m = Fe, Cr} {}
		   \left[  
                    y_i^{2a} y_j^{4f} y_k^{8i_1} y_l ^{8i_2} y_m ^{8j}
                   \right] H_{i j k l m} ^0 
		   + k_B T \sum_s \dfrac{m_s}{N} \sum_i y_i ^s \ln y_i ^s
	           \right]
  \end{split}
\end{align}

\noindent where the indexes $i, j, k, l, m$ run over the elements Fe and Cr for for the sublattices \siteA, \siteB,  \siteC, \siteD~and \siteE~respectively.
The compound index \textit{\ijklm} identifies a particular end member with a given configuration. 
$H_{i j k l m} ^0$ are the formation enthalpies at 0K of such configuraition, $m_s$ are the multiplicities of the sublattices \siteA \ldots \siteE~with $m_s = 2, 4, 8, 8, 8$ respectively, and $N$ is the total number of atoms in the unit cell.
As this is a well established method, we refer to the original literature for further details.
We calculate the formation enthalpies $\Delta H_{ijklm}^0$ in reference to the stable ground states of Fe and Cr, 

\begin{equation}
  \Delta H_{\ijklm}^0 = E_{\ijklm}^{\sigma} - \left( x_{\ijklm}^{Fe} E^{Fe}_{0} + x_{\ijklm}^{Cr} E^{Cr}_{0} \right)
\end{equation}

\noindent where $E_{\ijklm}^{\sigma}$  is the total energy of one unit cell of the \textsigma-phase, $E_{Fe}^{0}$ and $E_{Cr}^{0}$ are the energies per atom of Fe and Cr ground states, respectively, and $x_{\ijklm}^{Fe}$ and $x_{ijklm}^{Cr}$ are the atomic fractions of Fe and Cr in the \textsigma-phase unit cell for the specific configuration.

\paragraph{Partial occupancy}\label{methodology:po} To capture the impact of atomic disorder, we explicitly explore all possible atomic arrangements within the \textsigma-phase unit cell, incorporating experimentally reported partial occupancy of WS by Yakel~\cite{yakel_atom_1983} as reproduced in \autoref{fig:PartialOccupancies}.  
A random distribution in the different sublattices could be modelled for instance by a special quasirandom structure \MFnew{[SQS ref]} but  this would require a big unit cell to generate a meaningful representation of such atomic distributions, in particular due to the small Cr content in the  2\textit{a} site (< 0.5 atom per unit cell).
\MFnew{
  In principle, partial occupancy could also be modelled from Monte Carlo molecular dynamics simulation (MDMC). 
  First, such an approach would also require big unit cells for sufficient statitstics to represent the partial occupancies accurately. 
  Secondly, the size of those unit cells would also make it difficult to explore elastic constants from DFT. 
  Alternative approaches like MLIPs could be used to speed up the calculations. Such an alternative will be explored in future works. 

}

In order to keep the size of the simulation to one unit cell, we consider two approximations to the partial occupancy, in both cases disregarding Cr occupation in site \siteA~and rounding occupancy to integer numbers.  
A first model keeps full occupancy of Fe in the sites \siteA~and \siteD, full Cr occupancy in site \siteB, but sites \siteC~and \siteE~will have 5 Cr and 3 Fe atoms per unit cell.
We name this model PO1. 
In this condition, the total number of possible configurations is $1 \times 1 \times \binom{8}{3} \times 1 \times \binom{8}{3} = 3136 $, all with the same nominal composition and site occupancy.

A second model further transfers one Cr atom from \siteB~to \siteD.
We call this other model PO2.
The total number of possible configurations now explodes to $ 1 \times 4 \times \binom{8}{3}\times \binom{8}{1} \times \binom{8}{3} \approx 100k $.
We choose the explained naming convention according to increasing number of possible configurations.
In both models, the combinatorial explosion of the number of configurations hinders any extensive estimation of the elastic constants of a significant number of samples from DFT calculations, although in the second model the number of configurations is significantly reduced and approaching statistical significance would be easier.

\MFnew{
In comparison to these partial occupancy models, a sublattice model of the nominal composition of intersest would have a very different site occpancy. 
As shown in \autoref{fig:PartialOccupancies}, there are three possible configurations with nominal composition Fe\textsubscript{16}Cr\textsubscript{14} ($\approx$ 52at\% Fe). 
All of them have site \siteA~filled wih Cr instead of Fe. 
In addition, only one of the 8\textit{i}, 8\textit{i'} and 8\textit{j} sites would have Cr. 
This configuration might greatly destabilize the unit cell.
}

\paragraph{Elastic constants}

By designing deformation tensors which exploit the symmetry of the unit cell, the elastic tensor components can be determined from a parametrized strain-energy relation \cite{golesorkhtabar_elastic_2013}.



\begin{figure}[h!]
  
  \includegraphics[width=\linewidth]{Figure_Occupancies.pdf}
  \caption{\protect\label{fig:PartialOccupancies}
    Chemical composition of the \textsigma ~phase unit cell in all models with nominal composition Fe\textsubscript{16}Cr\textsubscript{14}. 
    The full sublattice models FO1, FO2 and FO3 are numbered according to increasing formation enthalpy. 
    Partial occupancy models PO1, PO2 are arranged according to the total number of possible configurations.
    Side markers indicate experimental partial occupancies.
  }
\end{figure}

Under the sublattice model, it is also customary to approximate any property of the disordered structure as a weighted average of the properties of the end-member configurations \MFnew{[missing ref]}.
In particular for the elastic constants $c_{IJ}$ 

\begin{equation}\label{cefcij}
  c_{IJ}^{CEF} (x, T) = \sum_{i j k l m = Fe, Cr} {}
  \left[  
    y_i^{2a} y_j^{4f} y_k^{8i_1} y_l ^{8i_2} y_m ^{8j}
\right] c_{IJ} ^{i j k l m}
\end{equation}

\paragraph{MLIP}

\MFnew{In recent years, universal force fields based on machine learning potentials have gained increasing attention. 
They are MLIPs fitted on massive DFT force-energy surface databases which have been progressively designed in order to improve the final  UFF accuracy. 
They allow for efficient use of computational resources to speed up calculations with high accuracy compared to DFT, and seem like a perfect tool for the task of scanning the high volume configurational space in the problem we present here.
However, the calculation of complex properties like the elastic tensor and the complex atomic environment in the \textsigma-phase may pose a challenge for this tools.
}

\vspace{1cm}
\MFnew{fine tuning or fitting as complementary tool}
\vspace{1cm}

Here we perform a combined DFT / machine learning interatomic potential (MLIP) approach to reduce the combinatorial sampling to the most relevant configurations, i.e. those with the lowest formation enthalpy.
As a first step, we select a subset of 25 representative configurations of each model and calculate their optimized unit cell and their elastic constants using DFT. 
We extend the resulting dataset to build a a specialized Fe-Cr \textsigma-phase DFT dataset consisting of 

\begin{itemize}
	\item the relaxation path of all the full-occupancy samples,
	\item the relaxation paths of the 50 ad-hoc chosen partial-occupancy samples,
	\item random deformations including isotropic compresion and expansions and atom-position shaking, 
	\item relaxation paths for unit cell optimization of the bcc, fcc, and hcp unary unit cells of the constitutents (Fe and Cr), 
	\item nearest neighbour expansions of the bcc, fcc, and hcp unary unit cells of the constituent elements (Fe and Cr)
\end{itemize}

All the available DFT data is used to fit a GRACE\MFnew{missing ref} potential from scratch. 



\section{Results}\label{sec:Results}

In the sublattice model, the Fe-Cr~\textsigma--phase exhibits a tetragonal unit cell in the space group P4\textsubscript{2}/mnm in the entire composition range.  
Our results show that partial occupancy disrupts unit cell symmetry, potentially affecting the mechanical response of the \textsigma--phase.  
Using \textit{spglib*}~\cite{Spglib_Togo} as implemented in the \textit{atomistic simulation environment}~\cite{HjorthLarsen_2017}, we find that even before full optimization of the unit cell, the only decoration of the lattice  sites according to the partial occupancy breaks the tetragonal symmetry of the sigma phase.  
\MFmarginpar{*}{}{would it be possible that occupancy in 2a recovers symmetry somehow?}
Upon decoration, all the configurations acquire a triclinic unit cell with space group \textit{P1}. 
The cell vector lengths and angles of the optimized triclinic cells are shown in \autoref{fig:SymmetryConsiderations} for all the configurations.  
Although small, the introduced structural variations might affect the mechanical properties of the \textsigma-phase. 

\paragraph{phase stability}\MFmarginpar{*}{}{include validation point}
The calculated convex hull for the full occupancy samples and both models with partial occupancy is shown in \autoref{subfig:FormationEnergies}. 
The formation enthalpies at 0K (DFT) of the \textsigma-phase are possitive for all compositions. 
However, CEF results show that the \textsigma-phase becomes stable starting at 873K, although the composition seems to be shifted towards the Cr rich region.
In the other hand, at the composition of interest, introducing partial occupancy stabilizes the phase as several samples have a formation enthalpy below the tie-line of the full-occupancy convex hull. 
In contrast, the samples at the same composition but within the sublattice model (FO1, FO2 and FO3)  lay tens of meV/at avobe the convex hull tie line.
No direct correlation between the formation enthalpy and unit cell distoritions is observed within the chosen samples.

The formation enthalpies of the PO1 set approaches the Gibbs free energies from the CEF at 0K, even for our limited selection of samples, and to the moment we dont find any sample with energies below this limit.
CEF site occupancey is alsor reported in comparison with the experimental data from Yakel ~\cite{yakel_atom_1983} in \autoref{subfig:CEF_Occupancy}.
We observe in general a very good agreement with experimental results.
At 873K we obtain nearly full Fe occupancy in sites \siteA~ and \siteD~ , both sites with 12-fold coordination.
However, experimental occupancy is about 10\% lower in both sites, which is compensated bay a small exess Fe in site \siteE .


\begin{figure}[t!]
  \subcaptionbox{\protect\label{subfig:angle_symmetry_rupture}}{\fbox{\includegraphics[height=4.5cm, trim=0.5cm -0.25cm 0.5cm -0.25cm, clip]{Figure_alfa_gamma_plots.pdf}}} %
  \subcaptionbox{\protect\label{subfig:lattice_parameters}}{\fbox{\includegraphics[height=4.5cm, trim=0.5cm -0.25cm 0.5cm -0.25cm, clip]{example-image-b}}} %
  \subcaptionbox{\protect\label{subfig:FormationEnergies}}{\includegraphics[height=5cm]{Figure_ConvexHull_gibbs.pdf}}
  \subcaptionbox{\protect\label{subfig:CEF_Occupancy}}{\includegraphics[height=5cm]{Figure_CEF_results_occupancy.pdf}}
  \caption{
    \protect\label{fig:SymmetryConsiderations}
    Effect of partial occupancy on unit cell geometry and stability. 
    (\subref{subfig:angle_symmetry_rupture}) Distribution of $\alpha$ and $\gamma$ angles for models with partial occupancy after full optimization with DFT. 
    (\subref{subfig:lattice_parameters}) Distribution of $\alpha$ and $\gamma$ angles for models with partial occupancy after full optimization with DFT. 
    (\subref{subfig:FormationEnergies}) Formation enthalpy at 0K for the full occupancy model and the two models with partial occupancy. 
    Gibbs free energy from the CEF model is displayed at 0K (black broken line) and 873K (blue broken line).
    Arrows indicate the formation enthalpies for the sublattice model samples with nominal composition 53at\%Fe. 
    (\subref{subfig:CEF_Occupancy}) Site occupacy profile as resulting from the minimization problem in equation \ref{eqn:CEF} at 873K (solid lines), 0K (broken lines) compared to experimental data (square markers), PO1 (rightwards arrows) and PO2 (leftwise arrows) occupancy. 
    Data at x\textsubscript{\textit{Fe}} = 0.52 was shifted horizontally to make all markers visible.
    }
\end{figure}


\paragraph{fitting of GRACE}
After this first the DFT calculations set (DFT0) we fit a GRACE \MFnewMissRef potential.
Here we intend to use this potential as a screening tool to scan the full configurational space and select the most relevant configurations for further DFT calculations. 
Therefore, the potential does not necesarilly need to be highly accurate. 
The DFT extended dataset is ilustrated in \autoref{fig:potential_dft_compare} together with a diagram of our workflow, while the final potential performance is reported in \autoref{tab:MLIPPerformance}.

%TODO: comment of all-config space evaluation. 
%TODO: comment on parity plot

\begin{figure}[t!]
  \subcaptionbox{\protect\label{subfig:extended_dft_dtaset}\MFnew{Missing HCP}}{\includegraphics[height=5cm]{Figure_DFTDataset.pdf}}
\subcaptionbox{\protect\label{subfig:dft_mlip_workflow}}{\includegraphics[height=5cm]{Figure_Workflow.pdf}}
%
\subcaptionbox{\protect\label{subfig:MLIPPrediction}}{
  		\includegraphics[height=5cm]{Figure_GRACE_potential_prediction.pdf}
	}
        \subcaptionbox{\protect\label{subfig:parityplot}}{
		\includegraphics[height=5cm]{Figure_ParityPlot.pdf}
	}
%%%
%%
\caption{
\protect\label{fig:potential_dft_compare}
%
(\subref{subfig:extended_dft_dtaset}) The energy - NN relationship for the extended DFT dataset used to train the GRACE potential.
Color scale indicates Fe composition of the samples, while the different markers indicate unit cell structure (bcc, fcc, hcp or \textsigma).
(\subref{subfig:dft_mlip_workflow}) Short chart showing the generalities of our workflow. 
The fitted GRACE potential is used only for screening samples from our convinatorial expansion of the partial occupancy models. 
GRACE predictions are also validated with subsequent DFT calculations. 
(\subref{subfig:MLIPPrediction}) Formation enthalpy ditribution as obtained from DFT and fitted GRACE pottential. 
(\subref{subfig:parityplot}) Parity between GRACE predictions and DFT calculations. 
Horizontal and vertical broken lines indicate the CEF formation enthalpy at 0K.
}
\end{figure}


\begin{table}

  \caption{\protect\label{tab:MLIPPerformance} Performance of the GRACE potential in approximating DFT energies, forces and stresses. 
		The performance is shown for the training set (Train) and a testing set (Test) and finally a validation set (Validation).
		The mean absolute error (MAE) and root mean square error (RMSE) are shown for each target variable.
	}
	\begin{tabular}{lrrrrrr}
		\toprule
		      & \multicolumn{2}{c}{Energy [meV/at]} & \multicolumn{2}{c}{Forces [eV/\AA]} & \multicolumn{2}{c}{Stresses [GPa]} \\
		      & MAE   & RMSE  & MAE   & RMSE  & MAE   & RMSE \\
		\midrule
	       Train &  5    &  30     &  0.005  & 0.01 & 0.008   & 0.03 \\
		Test & 10    &  90   &0. 01  & 0.02  & 0.01   & 0.025 \\
		Val   & \MFnew{Missing}   & \MFnew{Missing}   & \MFnew{Missing}  & \MFnew{Missing}  & \MFnew{Missing}  &  \MFnew{Missing}\\
		\bottomrule
	\end{tabular}%

\end{table}

\paragraph{Results on elastic constants}

%TODO: compare elastic constants to CEF
%TODO: estimation of errors


Samples from the sublattice model retain the tetragonal symmetry of the \textsigma-phase unit cell after full optimization, as expected. 
The elastic tensor of a tetragonal unit cell can be reduced to 6 independent components.
In Voigt notation~\cite{golesorkhtabar_elastic_2013}:

\begin{equation}\label{eqn:tetragonalElasticTensor}
  C^{tetragonal} =
  \begin{pmatrix}
    C_{11} & C_{12} & C_{13} & 0 & 0 & 0 \\
    C_{12} & C_{11} & C_{13} & 0 & 0 & 0 \\
    C_{13} & C_{13} & C_{33} & 0 & 0 & 0 \\
    0 & 0 & 0 & C_{44} & 0 & 0 \\
    0 & 0 & 0 & 0 & C_{44} & 0 \\
    0 & 0 & 0 & 0 & 0 & C_{66} \\
  \end{pmatrix}
\end{equation}

In the other hand, for the P1 space group of the PO1 and PO2 models, the elastic tensor has triclinic symmetry. 
Although such an elastic tensor would still be simmetric by definition, all the 21 components of the upper triangular half are in principle non-zero and independent from each other:

\begin{equation}\label{eqn:triclinicElasticTensor}
  C^{triclinic} =
  \begin{pmatrix}
    C_{11} & C_{12} & C_{13} & C_{14} & C_{15} & C_{16} \\
    C_{12} & C_{22} & C_{23} & C_{24} & C_{25} & C_{26} \\
    C_{13} & C_{23} & C_{33} & C_{34} & C_{35} & C_{36} \\
    C_{14} & C_{24} & C_{34} & C_{44} & C_{45} & C_{46} \\
    C_{15} & C_{25} & C_{35} & C_{45} & C_{55} & C_{56} \\
    C_{16} & C_{26} & C_{36} & C_{46} & C_{56} & C_{66} \\
  \end{pmatrix}
\end{equation}


In consequence, 21 parametrized deformation should be applied to solve such an elastic tensor. 
In addition, for each deformation tensor enough parameters should be applied as to determine each of the elastic tensor components after a least squares fit.
This makes the determination of the elastic tensor of the non-symmetrical triclinic unit cell unfeasible in the whole configurational space.

However, the size of ad-hoc selected PO1 and PO2 plus the validation set is manageable with DFT calculations.

We measure deviations in the elastic constants depending on the location of the components in the elastic tensor. 
For instance, there are several symmetry-based equalities on the tensor defined in equation \eqref{eqn:tetragonalElasticTensor}. 
For those components, define ratios, 

\begin{equation}\label{eqn:ratios}
    r_{IJ,KL} = \dfrac{C^{PO}_{IJ}}{C^{PO}_{KL}}, \qquad C^{CEF}_{IJ} = C^{CEF}_{KL} 
\end{equation}

\noindent such that $r_{11,22}$, $r_{44, 55}$ and $r_{13, 23}$ are defined. 

Then, for those elements which are different than zero in the tetragonal elastic tensor in equation \eqref{eqn:tetragonalElasticTensor}, we can define deviations in the triclinic tensor:

\begin{equation}\label{eqn:deviatiosnonzero}
\Delta^{\neq 0} _{IJ} = \dfrac{C_{IJ}^{PO} - C_{IJ}^{CEF}}{C_{IJ}^{CEF}}, \qquad C_{IJ}^{CEF} \neq 0
\end{equation}

Finally, for the set of elements which are zero in the tetragonal elastic tensor, we can take the value of the corresponding component of the triclinic elastic tensor as a direct measure of the deviation, 

\begin{equation}\label{eqn:deviationsarezero}
\Delta^{= 0} _{IJ} = C^{PO}_{IJ}/MPa, \qquad C^{CEF}_{IJ} = 0
\end{equation}
Notice that in all the definitions we take the elastic components from the CEF result in equation \eqref{cefcij} as reference.

 \begin{figure}[h!]
   \subcaptionbox{\protect\label{fig:ElasticTensorMains}}{
     \fbox{\includegraphics[height=5cm]{Figure_ratios_rijkl.pdf}}
   }
   \subcaptionbox{\protect\label{fig:ElasticTensorNozero}}{
     \fbox{\includegraphics[height=5cm]{Figure_Deviations_nonzero.pdf}}
   }

   \subcaptionbox{\protect\label{fig:ElasticTensorZero}}{
     \fbox{\includegraphics[height=5cm]{Figure_deviations_arezero.pdf}}
   }
  \caption{\protect\label{fig:ResultsElasticConstants} 
  Deviations in the elastic tensor components in the partial occupancy (PO) samples in comparison to the elastic tensor of the tetragonal lattice as a result from the CEF average from equation \eqref{cefcij}.
    (\subref{fig:ElasticTensorMains}) Ratios between PO components that are equal in CEF, \eqref{eqn:ratios}.
    (\subref{fig:ElasticTensorNozero}) relative change of the elastic tensor component between the PO samples and CEF average for non-zero valued constants in the latter, \eqref{eqn:deviatiosnonzero}. 
    (\subref{fig:ElasticTensorZero}) Values of the elastic tensor components where the corresponding components in the CEF result are zero.
  }

  %TODO: change elastic constants comparison to violin plots

\end{figure}\MFmarginpar{*}{}{Which FO sample?}


\paragraph{Elastic moduli including validation set}

%TODO: include isotropic elastic moduli from CEF
%TODO: include equations for isotropic

 In \autoref{fig:OccupancyEffect} we report the changes in the polycristalline elastic moduli\MFnewMissRef with nominal composition and fractional occupation of the sublattices and its relation to formation energy.
 First, the results for the FO samples clearly show the stiffening of the compound with increasing stability which can be expected in relation to deeper energy surfaces.
 We also see that the PO samples follow the same trend.
 A slight stiffening of the compound is already displayed for the small DFT dataset.

\begin{figure}[H]
  \fbox{ \includegraphics[width=\textwidth, trim=3.5cm 0cm 4cm 0cm, clip]{Figure_AverageElasticConstants.pdf}}
 \caption{\protect\label{fig:OccupancyEffect}
  Effect of occupancy on the stability and properties of the Fe\textsubscript{16}Cr\textsubscript{14} \textsigma ~phase.
 }
\end{figure}




%\begin{figure}
%
%  \caption{\protect\label{fig:MLIPPrediction}
%    (a) Comparison between DFT and MLIP formation energies for the Fe-Cr \textsigma~phase. 
%    The lower pannel shows the distribution of the DFT formation energies on the PO samples (1 and 2) from DFT.
%    Upper panel shows GRACE potential predictions for same PO samples. 
%    (b) Parity plot between DFT and GRACE predictions including FO and PO samples. 
%    The vertical line indicates the energy value of the convex hull tie line at Fe\textsubscript{16}Cr\textsubscript{14} composition.
%	}
%
%\end{figure}

\section{Conclusion}

In this work, we are presenting a methodical study of the effects of partial occupancy in the properties of the \textsigma phase of the Fe--Cr system.
For the first time, the properties of this phase are calculated taking into account the experimentally observed partial occupancies.
We demonstrate that partial occupancy significantly influences the mechanical response of the \textsigma-phase, which cannot be captured by the traditional full sublattice models.
Our results provide new insights into the mechanical behavior of the Fe--Cr \textsigma-phase and highlight the importance of considering partial occupancy in theoretical models.

\section{selection of working alloy}
final long term goal is CrMnFeCoNi (order by atomic number).
We select the easiest binary with $\sigma$ phase.
We select \fecrnominal because it was close to the experimental 50\%Fe and it is easy to distribute between the sublattices.
show experimental partial occupancies, and the modelled distributions.
(there are slides)
explaining breaking of symmetry by partial occupancies with figures.

\section*{Acknowledgements} 

This work is a part of the project ”Artificial Intelligence for Intermetallic Materials”–AIIM funded by the Deutsche Forschungsgemeinschaft (DFG No. 505643559). 

%\bibliographystyle{naturemag} 
\bibliography{main.bib}

\end{document}
