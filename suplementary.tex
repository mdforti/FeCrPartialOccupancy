\PassOptionsToPackage{unicode}{hyperref}
\PassOptionsToPackage{hyphens}{url}
%
\documentclass[superscriptaddress, 12pt]{revtex4-2}%twocolumn
\usepackage{amsmath,amssymb}
\usepackage{iftex}
% Use upquote if available, for straight quotes in verbatim environments
\usepackage{xcolor}
\usepackage{subcaption}
%\usepackage{longtable}
\usepackage{booktabs,array}
\usepackage{multirow}
\usepackage{calc} % for calculating minipage widths
% Correct order of tables after \paragraph or \subparagraph
\usepackage{etoolbox}
\usepackage{footnote}
\usepackage{graphicx}
\usepackage{bookmark}
\usepackage{babel}
\IfFileExists{xurl.sty}{\usepackage{xurl}}{} % add URL line breaks if available
\urlstyle{same}
\hypersetup{
  hidelinks,
  pdfcreator={LaTeX via pandoc}}
\usepackage{textgreek}
\usepackage{makecell}
\usepackage{multirow}
\usepackage{booktabs}
\date{}


\begin{document}

\title{Supplementary material: Influence of partial occupancy on the elastic tensor for FeCr \textsigma - phase.}

\author{Mariano Forti} \email{mariano.forti@icams.rub.de}
\affiliation{Interdisciplinary cetre for advanced materials simulations (ICAMS), Ruhr Universität Bochum, Germany}

\author{Thomas Hammerschmidt} \email{}
\affiliation{Interdisciplinary cetre for advanced materials simulations (ICAMS), Ruhr Universität Bochum, Germany}

\author{Guillaume Laplanche} \email{}
\affiliation{Ruhr Universität Bochum}


Following sections are not necessarily in order and need to be distributed
around the results/intro

\maketitle

\section{Results}

\subsection{Swarm plots of $C_{ij}$ deviations and correlation to formation energy}


\begin{figure}

   \subcaptionbox{\protect\label{fig:ElasticTensorMainsSW}}{
     \fbox{\includegraphics[height=5cm, trim = 0cm 0 3.5cm 1cm, clip]{Figure_C11_C13_C44_Deviations_on_equals.pdf}}
   }
   \subcaptionbox{\protect\label{fig:ElasticTensorNozeroSW}}{
     \fbox{\includegraphics[height=5cm, trim = 1.5cm 0 4cm 1cm, clip]{Figure_swarm_nonzero.pdf}}
   }
   \subcaptionbox{\protect\label{fig:ElasticTensorZeroSW}}{
     \fbox{\includegraphics[width=\linewidth, trim = 4cm 0cm 12cm 1cm, clip]{Figure_swarm_arezero.pdf}}
   }
  \caption{\protect\label{fig:ResultsElasticConstantsSW} 
  Deviations in the elastic tensor components in the partial occupancy (PO) samples in comparison to the elastic tensor of the tetragonal lattice as a result from the CEF average from equation \eqref{cefcij}.
    Formation energy of the samples is indicated in color code in units of eV/atom.
    (\subref{fig:ElasticTensorMains}) Ratios between PO components that are equal in CEF, \eqref{eqn:ratios}.
    (\subref{fig:ElasticTensorRatios}) relative change of the elastic tensor component between the PO samples and CEF average for non-zero valued constants in the latter, \eqref{eqn:deviatiosnonzero}. 
    Values of the elastic tensor components where the corresponding components in the CEF result are zero.
  }

\end{figure}

 \begin{figure}
%  \subcaptionbox{\protect\label{fig:ElasticTensorMains}}{
%  \fbox{\includegraphics[height=5cm, trim=0.5cm -1cm 1cm 0cm, clip]{Figure_C11_C13_C44_Deviations_on_equals.pdf}}
%}
%  \subcaptionbox{\protect\label{fig:ElasticTensorRatios}}{
%  \fbox{\includegraphics[height=5cm, trim=1.5cm 0cm 31cm -1cm, clip]{Figure_swarm_allrelations.pdf}}
%}
%
 % \subcaptionbox{\protect\label{fig:ElasticTensorRatios}}{
%  \fbox{\includegraphics[width=0.7\linewidth, trim=18cm 0cm 6cm 0cm, clip]{Figure_swarm_allrelations.pdf}}
%}
   \subcaptionbox{\protect\label{fig:DiffMainsCEF}}{
     \fbox{\includegraphics[height=5cm, width=0.3\textwidth]{example-image-a}}
   }
   \subcaptionbox{\protect\label{fig:ElasticTensorNozeroCEF}}{
     \fbox{\includegraphics[height=5cm, width=0.6\textwidth]{example-image-b}}
   }
   \subcaptionbox{\protect\label{fig:ElasticTensorZeroCEF}}{
     \fbox{\includegraphics[height=5cm, width=\textwidth]{example-image-c}}
   }
  \caption{\protect\label{fig:ResultsElasticConstants} 
    Comparison between elastic tensor components of the tetragonal, CEF average (CEF) model and the triclinic,
    partial occupancy (PO) model. 
    Formation energy of the samples is indicated in color code in units of eV/atom.
    () relative differences in the triclinic model for components that are equal in the tetragonal model. 
    () relative change of the elastic tensor component between the PO samples and FO sample, for non-zero valued constants in the triclinic cell (left). 
    For the constants that zero in the tetragonal cell, we present the relative change with respect to the largest constant of the tetragonal cell (right).
  }
\end{figure}


\end{document}
